\documentclass{article}

% if you need to pass options to natbib, use, e.g.:
%     \PassOptionsToPackage{numbers, compress}{natbib}
% before loading neurips_2018

% ready for submission
% \usepackage{neurips_2018}

% to compile a preprint version, e.g., for submission to arXiv, add add the
% [preprint] option:
%     \usepackage[preprint]{neurips_2018}

% to compile a camera-ready version, add the [final] option, e.g.:
     \usepackage[final]{neurips_2018}

% to avoid loading the natbib package, add option nonatbib:
%     \usepackage[nonatbib]{neurips_2018}

\usepackage[utf8]{inputenc} % allow utf-8 input
\usepackage[T1]{fontenc}    % use 8-bit T1 fonts
\usepackage{hyperref}       % hyperlinks
\usepackage{url}            % simple URL typesetting
\usepackage{booktabs}       % professional-quality tables
\usepackage{amsfonts}       % blackboard math symbols
\usepackage{nicefrac}       % compact symbols for 1/2, etc.
\usepackage{microtype}      % microtypography
\usepackage{amsmath}
\usepackage{amssymb}

\title{Optimization}

% The \author macro works with any number of authors. There are two commands
% used to separate the names and addresses of multiple authors: \And and \AND.
%
% Using \And between authors leaves it to LaTeX to determine where to break the
% lines. Using \AND forces a line break at that point. So, if LaTeX puts 3 of 4
% authors names on the first line, and the last on the second line, try using
% \AND instead of \And before the third author name.

\author{%
  David S.~Hippocampus\thanks{}\\%Use footnote for providing further information
    %about author (webpage, alternative address)---\emph{not} for acknowledging
    %funding agencies.} \\
  Department of Computer Science\\
  Cranberry-Lemon University\\
  Pittsburgh, PA 15213 \\
  \texttt{hippo@cs.cranberry-lemon.edu} \\
  % examples of more authors
  % \And
  % Coauthor \\
  % Affiliation \\
  % Address \\
  % \texttt{email} \\
  % \AND
  % Coauthor \\
  % Affiliation \\
  % Address \\
  % \texttt{email} \\
  % \And
  % Coauthor \\
  % Affiliation \\
  % Address \\
  % \texttt{email} \\
  % \And
  % Coauthor \\
  % Affiliation \\
  % Address \\
  % \texttt{email} \\
}

\begin{document}
% \nipsfinalcopy is no longer used

\maketitle

\begin{abstract}
Optimization
\end{abstract}

\section{Lagrange Dual}
Suppose we want to optimize $x\in \mathbb{R}^n$:
\begin{eqnarray*}
\min_x f(x)\\
s.t.\qquad c_i(x) \le 0\\
h_j(x) = 0
\end{eqnarray*}
We introduce the generalized Lagrange function:
\begin{equation}
L(x,\alpha,\beta)=f(x)+\sum_i\alpha_ic_i(x)+\sum_j\beta_jh_j(x) \qquad \alpha_i \ge 0
\end{equation}
Then the \textbf{primal function}:
\begin{equation}
\theta_P(x) = \max_{\alpha,\beta:\alpha_i\ge0}L(x,\alpha,\beta)
\end{equation}
\textbf{Important Properties}: 
\begin{enumerate}
\item if the condition of x does not satisfy, i.e., $c_i(x)>0$ or $h_j(x)\neq0$, then we have
\begin{equation*}
\theta_P(x)=+\infty
\end{equation*}
\item if x satisfies, we have
\begin{equation*}
\theta_P(x)=f(x)
\end{equation*}
\end{enumerate}
We define $p^*$ as the solution of the primal problem:
\begin{equation}
p^*=\min_x\theta_P(x)
\end{equation}

\subsection{Dual Problem}
We define a function of $\alpha, \beta$:
\begin{equation}
\theta_D(\alpha, \beta)=\min_xL(x,\alpha,\beta)
\end{equation}
We define the dual problem as:
\begin{equation}
d^*=\max_{\alpha,\beta:\alpha_i\ge0}\theta_D(\alpha,\beta)=\max_{\alpha,\beta:\alpha_i\ge0}\min_xL(x,\alpha,\beta)
\end{equation}

\subsection{Properties}
If both $p^*$ and $d^*$ exists, we have:
\begin{equation}
d^* \le p^*
\end{equation}
\textbf{KKT} condition, if following conditions hold, we have $d^*=p^*$:
\begin{enumerate}
\item $f(x)$ and inequality condition $c_i(x)$ both convex;
\item Equality condition $h_i(x)$ is an affine transform;
\item Exist $x$ satisfies $c_i(x)<0$
\end{enumerate}

\textbf{KKT} condition: if $\{x^*,\alpha,\beta\}$ with $x^*$ as a local optimum, we have
\begin{enumerate}
\item Stationary:
\begin{equation}
\nabla f(x^*)+\sum_i\alpha_i\nabla c_i(x)+\sum_j\beta_j\nabla h_j(x)=0
\end{equation}
\item Primal feasibility:
\begin{eqnarray}
c_i(x^*)\le 0\\
h_j(x^*)=0
\end{eqnarray}
\item Dual feasibility:
\begin{equation}
\alpha_i\ge0
\end{equation}
\item Complementary slackness:
\begin{equation}
\alpha_ic_i(x^*)=0
\end{equation}
\end{enumerate}

\subsection{Application 1: SVM}
\begin{eqnarray}
\min_{w,b}\frac{1}{2}||w||^2\\
s.t.\qquad (w^Tx_i+b)y_i\ge1
\end{eqnarray}
The function with Lagrange multiplier is:
\begin{equation*}
L(w,b;\alpha) =\frac{1}{2}||w||^2+\sum_{i=1}^n\alpha_i(1-y_i(wx_i+b))
\end{equation*}

Stability:
\begin{eqnarray}
\nabla_w L= w-\sum_{i=1}^n\alpha_iy_ix_i=0 \Rightarrow w=\sum_{i=1}^n\alpha_iy_ix_i\\
\nabla_b L=0 \Rightarrow \sum_{i=1}^n\alpha_iy_i=0
\end{eqnarray}
Then, make the substitution for $w$, we have:
\begin{eqnarray*}
L(\alpha)&=&\frac{1}{2}\sum_i\sum_j\alpha_i\alpha_jy_iy_jx^T_ix_j+\sum_i\alpha_i-\sum_i\sum_j\alpha_i\alpha_jy_iy_jx^T_ix_j-b(\sum_i\alpha_iy_i)\\
&=&\sum_i\alpha_i-\frac{1}{2}\sum_i\sum_j\alpha_i\alpha_jy_iy_jx^T_ix_j \qquad s.t. \qquad \sum_{i=1}^n\alpha_iy_i=0
\end{eqnarray*}
We will maximize the dual subject to the constraint.

\end{document}
